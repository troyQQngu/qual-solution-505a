\documentclass{article}
\usepackage[utf8]{inputenc}
\usepackage[english]{babel}
\usepackage{xcolor}
\usepackage{amsmath}
\usepackage{bbm}
\usepackage[]{amsthm} %lets us use \begin{proof}
\usepackage[]{amssymb} %gives us the character \varnothing
\usepackage[a4paper, total={6in, 8in}]{geometry}
\newcommand{\prob}{\mathbb{P}}
\newcommand{\E}{\mathbb{E}}
\newcommand{\sol}{\\\\\textit{Solution. }}
\def\beq#1\eeq{%
    \begin{equation*}\begin{split}%
      #1%  
    \end{split}\end{equation*}%
}

\title{MATH 505a Fall 2017 Qual Solution Attempts}
\author{Troy Tao}
\date\today

\begin{document}
\maketitle 
Contact \textcolor{blue}{yntao@usc.edu} if you think this document needs revision.


\section*{Problem 1}
Let $X$ be uniform on [1,5], let $Y$ be uniform on [0,1], and assume that $X$ and $Y$ are independent.
\subsection*{(a)}
Compute the probability density function of the product $XY$.
\color{blue}
\sol \beq
\prob(XY\leq t) &= \prob(Y\leq \frac{t}{X})\\
&= \int \prob(Y\leq \frac{t}{x})f_X(x)dx
\eeq 
\textit{Case 1.} $0<t\leq 1$,
\beq
\prob(Y\leq \frac{t}{X}) &=\int_1^5 \frac{t}{x}\frac{1}{4}dx\\
&=\frac{t}{4}\ln(5)
\eeq 
\textit{Case 2. }$1<t\leq 5$,
\beq
    \prob(Y\leq \frac{t}{X}) &= \int_1^t 1\cdot\frac{1}{4}dx+\int_t^5\frac{t}{x}\cdot \frac{1}{4}dx\\
    &=\frac{1}{4}(t-1)+\frac{t}{4}\ln(\frac{5}{t})
\eeq 
Therefore, by differentiating:
\beq
    f_{XY}(t) = \begin{cases}
    \frac{\ln(5)}{4}, & t\in(0,1],\\
    \frac{1}{4}\ln\left(\frac{5}{t}\right), & t\in(1,5],\\
    0, & \text{otherwise.}
    \end{cases}
\eeq
\color{black}
\subsection*{(b)}
Compute the cumulative distribution function of the ratio $X/Y$.
\color{blue}
\sol 
\beq
\prob(X/Y\leq t) &=\prob(Y\geq X/t)\\
&=\int\prob(Y\geq x/t)f_X(x)dx
\eeq 
\textit{Case 1. }$1<t\leq 5$,
\beq
    \prob(Y\geq X/t) &=\int_1^t(1-\frac{x}{t})\cdot \frac{1}{4}dx\\
    &=\frac{t}{8}+\frac{1}{8t}-\frac{1}{4}
\eeq 
\textit{Case 2. }$t\geq 5$,
\beq
    \prob(Y\geq X/t) &= \int_1^5(1-\frac{x}{t})\cdot \frac{1}{4}dx\\
    &=1-\frac{3}{t}
\eeq 
\color{black}
\subsection*{(c)}
Compute the characteristic function of the sum $X+Y$.
\color{blue}
\sol
\beq
    \E(e^{it(X+Y)})&= \E(e^{itX})\E(e^{itY})\\
    &=\int_1^5e^{itx}\frac{1}{4}dx\int_0^1e^{ity}dy\\
    &=-\frac{1}{4t^2}(e^{5it}-e^{it})(e^{it}-1)
\eeq 
\color{black}
\subsection*{(d)}
Compute the moment generating function of the random variable $X-\ln(Y)$.
\color{blue}
\sol
\beq
    \E(e^{t(X-\ln Y)}) &=\E(e^{tX}\cdot e^{-t\ln Y})\\
    &=\E(e^{tX})\cdot \E(Y^{-t})\\
    &=\int_1^5e^{tx}\frac{1}{4}dx\int_0^1y^{-t}dy\\
\eeq
Notice that the right multiplicand's integrability depends on $t$, so
$$\E(e^{t(X-\ln Y)}) = \begin{cases}
    \frac{e^{5t}-e^t}{4t(1-t)}, & t< 1,\\
    \infty, & t\geq1
    \end{cases}$$
\color{black}
\section*{Problem 2}
An urn contains $2n$ balls, coming in pairs: two balls are labeled ``1'', two balls are labeled ``2'',..., two balls are labeled ``n''. A sample of size $n$ is taken without replacement. Denote by $N$ the number of pairs in the sample. Compute the expected value and the variance of $N$. \textbf{You do not need to simplify the expression for the variance.}
\color{blue}
\sol Let $X_i$ be the indicator function of the pair of balls labeled ``i'' are selected. And the probability of any pair being selected is the ratio of the number of combinations to select $n-2$ balls from the rest of $2n-2$ balls and the total number of combinations to select $n$ balls from $2n$ balls.
\beq
    \E(N) &= \E(\sum_{i=1}^n X_i)\\
    &=\sum_{i=1}^n  \prob(X_i=1)\\
    &= n\cdot \frac{{2n-2 \choose n-2}}{{2n \choose n}}\\
    &= \frac{n(n-1)}{2(2n-1)}
\eeq
\beq
    \E(X_i^2) &= \E(X_i)\\
    &= \frac{n-1}{2(2n-1)}
\eeq 
Notice that the probability of two pairs being selected is the ratio of the number of combinations to select $n-4$ balls from the rest of $2n-4$ balls and the total number of combinations to select $n$ balls from $2n$ balls. so for $i \neq j$,
\beq
    \E(X_iX_j)&=\prob(X_i=1,\ X_j=1)\\
        &= \frac{{2n-4 \choose n-4}}{{2n  \choose n}}\\
        &= \frac{n(n-1)(n-2)(n-3)}{2n(2n-1)(2n-2)(2n-3)}
\eeq
\beq
    Var(N) &= \E(N^2)-\E(N)^2\\
        &= \sum_{i=1}^n \E(X_i^2)+\sum_{i\neq j}^n\E(X_iX_j) -\E(N)^2\\
        &= n\cdot\frac{(n-1)}{2(2n-1)}+n(n-1)\cdot\frac{n(n-1)(n-2)(n-3)}{2n(2n-1)(2n-2)(2n-3)}-\left(\frac{n(n-1)}{2(2n-1)}\right)^2
\eeq 
\color{black}

\section*{Problem 3}
Let $U_1,U_2,...$ be iid random variables, uniformly distributed on [0,1], and let $N$ be a Poisson random variable with mean value equal to one. Assume that $N$ is independent of $U_1,U_2,...$ and define
$$Y=\begin{cases}0, &\text{if } N=0,\\
\max_{1\leq i \leq N}U_i, &\text{if } N>0.
\end{cases}$$
Compute the expected value of $Y$.
\color{blue}
\sol First we compute the expectation of $max_{1\leq i\leq k}U_i$ for some $k\geq 1$. For $0<t<1$,
\beq
    \prob(\max_{1\leq i \leq k}U_i\leq t) &= \prod_{i=1}^k  \prob(U_i\leq t)\\
        &=t^k
\eeq 
Since $U_i$ only takes nonnegative value,
\beq
    \E(\max_{1\leq i\leq k}U_i) &= \int_0^1(1-t^k)dt\\
        &=\frac{k}{k+1}
\eeq 
We compute the expectation of $Y$ by conditioning on $N$,
\beq
    \E(Y) &= \sum_{k=0}^\infty \E(Y\vert N=k)\prob(N=k)\\
        &=0+\sum_{k=1}^\infty\frac{k}{k+1}e^{-1}\frac{1}{k!}\\
        &=e^{-1}\sum_{k=1}^\infty\left(\frac{1}{k!}-\frac{1}{(k+1)!}\right)\\ 
        &=e^{-1}
\eeq 
\end{document}
