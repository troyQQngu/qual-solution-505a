\documentclass{article}
\usepackage[utf8]{inputenc}
\usepackage[english]{babel}
\usepackage{xcolor}
\usepackage{amsmath}
\usepackage{bbm}
\usepackage[]{amsthm} %lets us use \begin{proof}
\usepackage[]{amssymb} %gives us the character \varnothing
\usepackage[a4paper, total={6in, 8in}]{geometry}

\title{MATH 505a Spring 2022 Qual Solution Attempts}
\author{Troy Tao}
\date\today

\begin{document}
\maketitle 
Contact \textcolor{blue}{yntao@usc.edu} if you think this document needs revision.

\section*{Problem 1}
\subsection*{(a)}
Let $X_1,X_2,X_3$ be independent exponential random variables with parameter $\lambda=1$. So $\mathbb{P}(X_i>x)=e^{-x},x>0$. Find
$$
    \mathbb{E}\left(\frac{X_1}{X_1+X_2+X_3}\right).
$$
\textcolor{blue}{
    \textit{Solution.} Let $Y = X_1+X_2$, then compute the pdf of $Y$: 
    \begin{equation*}
        \begin{split}
            \mathbb{P}(X_2+X_3\leq y) &=\int_0^y\mathbb{P}(X_2 \leq y-x)f_{X_3}(x)dx\\
            & = \int_0^y (1-e^{-(y-x)})e^{-x}dx\\
            & = \left[-e^{-x}-xe^{-x}\right]_0^y\\
            & = 1-e^{-y}(y+1), \ y \geq 0
        \end{split}
    \end{equation*}
    So by differentiating the cdf, we have
    \begin{equation*}
        f_Y(y) = ye^{-y}, \  y \geq 0
    \end{equation*}
    Now consider the probability:
    \begin{equation*}
        \begin{split}
            \mathbb{P}\left(\frac{X_1}{X_1+Y} \leq z\right) &= \mathbb{P}\left(X_1 \leq \frac{zY}{1-z}\right)\\
                &=\int_0^{\infty}\mathbb{P}\left(X_1 \leq \frac{zy}{1-z}\right)f_Y(y)dy\\
                &=\int_0^{\infty}\left[1-exp\left(-\frac{zy}{1-z}\right)\right]\cdot y \cdot exp(-y)dy\\
                &=1-(1-z)^2, \ 0\leq z \leq 1
        \end{split}
    \end{equation*}
    And from the fact that {\large$\frac{X_1}{X_1+Y}\geq 0$} (only takes non-negative value), we can compute the expectation by the complementary cdf:
    \begin{equation*}
        \mathbb{E}\left(\frac{X_1}{X_1+X_2+X_3}\right)=\int_0^1(1-z)^2dz = \frac{1}{3} 
    \end{equation*}
}
\subsection*{(b)}
Let $(X,Y)$ be independent uniforms on $[0,1]$. Find the joint density function of $X$ and $V=X+Y$. Find $f(x|v)$, the density function of $X$ conditional on $V = v$. Also, find $\mathbb{E}(X|V)$. 
\textcolor{blue}{
\begin{proof}
    First we should compute the pdf of $V$. When $0<v<1$: 
    \begin{equation*}
        \mathbb{P}(X+Y\leq v) = \int_0^v(v-y)dy = \frac{1}{2}v^2
    \end{equation*}
    When $1 \leq v <2$:
    \begin{equation*}
        \mathbb{P}(X+Y\leq v) = \int_{v-1}^1(v-y)dy+(v-1) = -\frac{1}{2}v^2+2v-1
    \end{equation*}
    So we have the pdf:
    \begin{equation*}
        f_V(v) = 
        \begin{cases} 
          v & 0<v<1 \\
          2-v & 1\leq v <2 
       \end{cases}
    \end{equation*}
    Note that the conditional pdf is
    \begin{equation*}
        f_{X|V}(x|v) = \frac{f_{X,V}(x,v)}{f_V(v)} = \frac{f_X(x)f_Y(y)}{f_V(v)}, \  y=v-x
    \end{equation*}
    plug in the previous results,
    \begin{equation*}
        f_{X|V}(x|v) =
        \begin{cases}
          \frac{1}{v} & 0<x<v<1\\
          \frac{1}{2-v} & 1 \leq v <2, \  v-1< x < 1
        \end{cases}
    \end{equation*}
    So the expectation follows:
    \begin{equation*}
    \mathbb{E}(X|V) = \frac{1}{2} V \cdot \mathbbm{1}_{(0,1)}(V) + \frac{2V-V^2}{4-2V}\cdot \mathbbm{1}_{[1,2)}(V)
    \end{equation*}
\end{proof}
}
\section*{Problem 2}
In an election, candidates A receives $n$ votes, and candidate B receives $m$ votes, where $n>m$. Assuming that all {\LARGE$\genfrac(){0pt}{2}{n+m}{m}$} orderings are equally likely, show that the probability that A is always ahead in the count of votes is $(n-m)/(n+m)$.
\textcolor{blue}{
\begin{proof}
   Denote $P_{i,j} = \mathbb{P}$(A is always ahead $|$ A received $i$ votes, B received $j$ votes). By conditioning on the last vote, we have
   \begin{equation*}
   \begin{split}
        P_{n,m} &= \mathbb{P}(\text{A received the last vote})P_{n-1,m}+\mathbb{P}(\text{B received the last vote})P_{n,m-1}\\
                &= \frac{n}{n+m}P_{n-1,m}+\frac{m}{n+m} P_{n,m-1}
   \end{split}
   \end{equation*}
   Then we construct an induction: $\forall k \in \mathbb{N}$, if $n+m = k$ and $n>m$, then $P_{n,m}=\frac{n-m}{n+m}$.\\
   \begin{enumerate}
       \item $k = 1,\ n>m \implies n=1, \ m=0$ and $P_{1,0} = 1$\\
       \item Given that the statement is true for $n+m=k$, let $n+m = k+1, \ n>m$. Then we have
            \begin{equation*}
                P_{n,m} = \frac{n}{n+m}P_{n-1,m}+\frac{m}{n+m} P_{n,m-1}
            \end{equation*}
            Note that $n+m-1=k \implies P_{n,m-1} = \frac{n-m+1}{n+m-1}$ and $P_{n-1,m}=\frac{n-m-1}{n+m-1}$. So
            \begin{equation*}
            \begin{split}
                P_{n,m} &= \frac{n}{n+m}\frac{n-1-m}{n-1+m}+\frac{m}{n+m}\frac{n-m+1}{n+m-1}\\
                        &= \frac{(n-1+m)(n-m)}{(n-1+m)(n+m)}\\
                        &= \frac{n-m}{n+m}
            \end{split}
            \end{equation*}
            In the case that $n=m+1$, $P_{n-1,m}=0$ since B eventually will catch up A. Therefore, although it's not in our assumption, the equation $P_{n-1,m}=0=(n-1-m)/(n+m-1)$ is still true.
   \end{enumerate}
\end{proof}
}

\section*{Problem 3}
Let $n$ be a positive integer with prime factorization $ n=p_1^{m_1} \cdots p_k^{m_k}$ for distinct primes $p_1,\cdots ,p_k$ with $m_1,\cdots,m_k>0$. Choose an integer $N$ uniformly at random from the set $\{1,2,\cdots,n\}.$. Show that the probability that $N$ shares no common prime factor with n is equal to
\begin{equation*}
    \left(1-\frac{1}{p_1}\right)\left(1-\frac{1}{p_2}\right)\cdots \left(1-\frac{1}{p_k}\right).
\end{equation*}
(Hint: use inclusion-exclusion)
\textcolor{blue}{
\begin{proof}
    Let $A_i = \{N = ap_i, a \in \mathbb{N}\}$ and for $I \subset \{1,2,\cdots,n\}$
    \begin{equation*}
        \begin{split}
            \mathbb{P}(\cap_{i \in I} A_i) &= \mathbb{P}(N = a\prod_{i\in I}p_i,\ a \in \mathbb{N})\\
            &= \frac{\frac{n}{\prod_{i\in I}p_i}}{n}\\
            &= \prod_{i\in I}\frac{1}{p_i}
        \end{split}
    \end{equation*}
    Now by inclusion-exclusion theorem,
    \begin{equation*}
        \begin{split}
            \mathbb{P}(\text{N is co-prime to n}) &= 1-\mathbb{P}(\cup_{i=1}^{k} A_i)\\
            &= 1-\left(\sum_{i=1}^k\frac{1}{p_i}-\sum_{i,j=1}^k\frac{1}{p_ip_j}+\cdots+(-1)^{k+1}\frac{1}{\prod_{i=1}^kp_i} \right)\\
            &= 1+\sum_{i=1}^k\left(-\frac{1}{p_i}\right)+\sum_{i,j=1}^k\left(-\frac{1}{p_i} \right)\left( -\frac{1}{p_j}\right)+\cdots+\prod_{i=1}^k\left(-\frac{1}{p_i}\right)\\
            &= \left(1-\frac{1}{p_1}\right)\left(1-\frac{1}{p_2}\right)\cdots \left(1-\frac{1}{p_k}\right)
        \end{split}
    \end{equation*}
\end{proof}
}

\end{document}