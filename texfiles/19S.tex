\documentclass{article}
\usepackage[utf8]{inputenc}
\usepackage[english]{babel}
\usepackage{xcolor}
\usepackage{amsmath}
\usepackage{bbm}
\usepackage[]{amsthm} %lets us use \begin{proof}
\usepackage[]{amssymb} %gives us the character \varnothing
\usepackage[a4paper, total={6in, 8in}]{geometry}
\newcommand{\prob}{\mathbb{P}}
\newcommand{\E}{\mathbb{E}}

\title{MATH 505a Spring 2019 Qual Solution Attempts}
\author{Troy Tao}
\date\today

\begin{document}
\maketitle 
Contact \textcolor{blue}{yntao@usc.edu} if you think this document needs revision.


\section*{Problem 1}
Suppose that each of 5 jobs is assigned at random to one of three servers A,B and C. [For example, one possible outcome would be that job 1 goes to server B, job 2 goes to server C, job 3 goes to server C, job 4 goes to server B and job 5 goes to server A. ``At random'' here means that there are $3^5$ equally likely outcomes]
\subsection*{(a)}
Find the probability that server C gets all 5 jobs.
\color{blue}
$$\prob(\text{C gets 5 jobs})=\left(\frac{1}{3}\right)^5 $$
\color{black}
\subsection*{(b)}
Let $S$ be the number of servers that get exactly one job. Find $\E S$.
\color{blue}
\\\\\textit{Solution. }
Let $I_A,I_B,I_C$ denote the indicator functions of $A,B,C$ get exactly one job, respectively. Then,
\begin{equation*}
    \begin{split}
        \E(S) &= \E(I_A+I_B+I_C)\\
        &=\prob(I_A=1)+\prob(I_B=1)+\prob(I_C=1)\\
        &= 3{5 \choose 1}\left(\frac{1}{3}\right)\left(\frac{2}{3}\right)^4\\
        &= \frac{80}{81}
    \end{split}
\end{equation*}
\color{black}
\subsection*{(c)}
Find the probability that no server gets more than 2 jobs.
\color{blue}
\begin{equation*}
    \begin{split}
        \prob(\text{no server gets more than 2 jobs}) &= \prob(\text{2 servers get 2 jobs each, 1 server get 1 job})\\
        &= {3 \choose 1} \left(\frac{1}{3}\right)^2\left(\frac{1}{3}\right)^2\left(\frac{1}{3}\right)\\
        &= 3 \left(\frac{1}{3}\right)^5\\
        &= \frac{1}{81}
    \end{split}
\end{equation*}
\color{black}
\subsection*{(d)}
Take the same story, but with $m$ in pace of 5 for the number of jobs, and $n$ in place of 3 for the number of servers. Find the variance of $S$, in terms of $m$ and $n$.
\color{blue}
\\\\\textit{Solution. }Let $I_i$ denotes the indicator function that the $i$-th server gets exactly 1 job.
\begin{equation*}
    \begin{split}
        Var(S) &= \E(S^2)-(\E S)^2\\
        &= \E(\sum_{i=1}^nI_i^2+2\sum_{i\neq j}^nI_i\cdot I_j)-\left(\sum_{i=1}^n\E(I_i)\right)^2\\
        &= \sum_{i=1}^n\prob(I_i=1)+2\sum_{i\neq j}^n\prob(I_i=1,I_j=1)-\left(\sum_{i=1}^n\prob(I_i=1)\right)^2\\
        &= m\left(\frac{n-1}{n}\right)^{m-1}-\left(m\left(\frac{n-1}{n}\right)^{m-1}\right)^2+2\cdot n(n-1)\cdot m(m-1)\cdot\left(\frac{1}{n}\right)^2 \left(\frac{n-2}{n}\right)^{m-2}
    \end{split}
\end{equation*}
\color{black}
\section*{Problem 2}
\subsection*{(a)}
Suppose that $X$ is Poisson with parameter $\lambda$. Find the characteristic function of $X$.
\color{blue}
\begin{equation*}
    \begin{split}
        \E(e^{itX}) &= \sum_{k=0}^\infty e^{itk}e^{-\lambda}\frac{\lambda^k}{k!}\\
        &=e^{-\lambda(1-e^{it})}\sum_{k=0}^\infty e^{-e^{it}\lambda}\frac{(e^{it}\lambda)^k}{k!}\\
        &=e^{-\lambda(1-e^{it})}
    \end{split}
\end{equation*}
\color{black}
\subsection*{(b)}
Suppose that $X_n$ is Poisson with parameter $\lambda_n$ and that $\lambda_n \rightarrow \infty$. Show using characteristic functions that $(X_n-\lambda_n)/\sqrt{\lambda}$ converges in distribution, and describe the limiting distribution.
\color{blue}
\begin{proof}
\begin{equation*}
    \begin{split}
        \lim_{n \rightarrow \infty}\E(e^{it(X_n-\lambda_n)/\sqrt{\lambda}}) &= \lim_{n \rightarrow \infty} \exp{(-it\sqrt{\lambda_n}-\lambda_n+\lambda_n e^{i\frac{t}{\sqrt{\lambda_n}}})}\\
        &=\lim_{n \rightarrow \infty}\exp{\left(-it\sqrt{\lambda_n}-\lambda_n+\lambda_n \sum_{k=0}^\infty\frac{(it/\sqrt{\lambda_n})^k}{k!}\right)}\\
        &=\lim_{n \rightarrow \infty}\exp{\left(-it\sqrt{\lambda_n}+\lambda_n(it/\sqrt{\lambda_n})+\lambda_n \frac{(it/\sqrt{\lambda_n})^2}{2}+\lambda_n \frac{(it/\sqrt{\lambda_n})^3}{6}+\cdots\right)}\\
        &=\lim_{n \rightarrow \infty}\exp{\left(-\frac{t^2}{2}- \frac{it^3}{6\lambda^{1/2}}+\cdots\right)}\\
        &=\exp{\left(-\frac{t^2}{2}\right)}
    \end{split}
\end{equation*}
It converges to the standard normal distribution.
\end{proof}
\color{black}
\section*{Problem 3}
A stick of length 1 is broken into two pieces at a uniformly distributed random point.
\subsection*{(a)}
Find the expected length of the smaller piece.
\color{blue}
\\\\\textit{Solution. }Let $X_1,X_2,U_1$ denote the length of the smaller stick, the length of the larger stick, and the location of the first break point, respectively.
\begin{equation*}
    \begin{split}
        F_{X_1}(x) &= \prob(X_1\leq x, U_1\leq 1/2) + \prob(X_1\leq x, U_1> 1/2)\\
        &= x+ (1-(1-x))\\
        &= 2x\\
    \end{split}
\end{equation*}
$$f_{X_1}(x) = 2,\ 0<x<1/2$$
Similarly, we have
$$F_{X_2}(x) = 1-\prob(X_1<1-x) = 2x-1 $$
$$f_{X_2}(x) = 2,\ 1/2<x<1$$
From the pdf of $X_1$, we have,
$$\E(X_1)=\frac{1}{4}$$
\color{black}
\subsection*{(b)}
Find the expected value of the ratio of the smaller length over the larger.
\color{blue}
\\\\\textit{Solution. }
\begin{equation*}
    \begin{split}
        \prob(\frac{X_1}{X_2}\leq t) &=\prob(\frac{X_1}{1-X_1}\leq t)\\
        &= \prob(X_1\leq \frac{t}{t+1})\\
        &= \frac{2t}{t+1},\ 0<t<1
    \end{split}
\end{equation*}
Since the ratio only takes non-negative value, we can use complementary cdf to compute expectation:
\begin{equation*}
    \begin{split}
        \E\left(\frac{X_1}{X_2}\right) &= \int_0^1 1-\frac{2t}{1+t}dt\\
        &=1-2+2\int_0^1\frac{1}{1+t}dt\\
        &=-1+2\ln(2)
    \end{split}
\end{equation*}
\color{black}
\subsection*{(c)}
Suppose the larger piece is then broken at a random point, uniformly distributed over its length, independent of the first break point. There are then three pieces. Find the probability the longest of the three has length more than 1/2.
\color{blue}
\\\\\textit{Solution. }Let $X_3, U_2$ be the length of the larger piece of the previous larger piece and the second break point location (start from the left end of the $X_2$). First notice that the pdf of $U_2$:
$$F_{U_2|X_2=x}(t) = \frac{t}{x}$$
Since $\prob(X_1>1/2)=0$, so the desired probability is
\begin{equation*}
    \begin{split}
        \prob(X_3>1/2) &= \int_{1/2}^1\prob(X_3>1/2,U_2>x/2)+\prob(X_3>1/2,U_2\leq x/2)\cdot f_{X_2}(x)dx\\
        &= \int_{1/2}^1[\prob(U_2>1/2\vert X_2=x)+\prob(U_2<X_2-1/2\vert X_2=x)\cdot 2 dx\\
        &= 2 \int_{1/2}^11-\frac{1}{2x}+\frac{1}{x}(x- \frac{1}{2}) dx\\
        &= 2+2\ln(1/2)
    \end{split}
\end{equation*}
\end{document}
