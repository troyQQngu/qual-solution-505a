\documentclass{article}
\usepackage[utf8]{inputenc}
\usepackage[english]{babel}
\usepackage{xcolor}
\usepackage{amsmath}
\usepackage{bbm}
\usepackage[]{amsthm} %lets us use \begin{proof}
\usepackage[]{amssymb} %gives us the character \varnothing
\usepackage[a4paper, total={6in, 8in}]{geometry}
\newcommand{\prob}{\mathbb{P}}
\newcommand{\E}{\mathbb{E}}

\title{MATH 505a Spring 2020 Qual Solution Attempts}
\author{Troy Tao}
\date\today

\begin{document}
\maketitle 
Contact \textcolor{blue}{yntao@usc.edu} if you think this document needs revision.


\section*{Problem 1}
Each pack of bubble gum contains one of $n$ types of coupon, equally likely to be each of the types, independently from one pack to another. Let $T_j$ be number of packs you must buy to obtain coupons of $j$ different types. Note that $T_1=1$ always.
\subsection*{(a)}
Find the distribution and expected value of $T_2-T_1$ and of $T_3-T_2$.
\color{blue}
\\\\\textit{Solution. }Notice that these differences are geometric distributions:
\begin{equation*}
    \prob(T_2-T_1=t) = \left(\frac{n-1}{n}\right)\left(\frac{1}{n}\right)^{t-1},\ t \geq 1.
\end{equation*}
\begin{equation*}
    \prob(T_3-T_2=t) = \left(\frac{n-2}{n}\right)\left(\frac{2}{n}\right)^{t-1},\ t \geq 1.
\end{equation*}
\begin{equation*}
    \E(T_2-T_1) = \frac{n}{n-1}    
\end{equation*}
\begin{equation*}
    \E(T_3-T_2) = \frac{n}{n-2}
\end{equation*}
\color{black}
\subsection*{(b)}
Compute $\E T_n$.
\color{blue}
\\\\\textit{Solution. }
\begin{equation*}
    \begin{split}
        \E(T_n) &= \E [(T_n-T_{n-1})+(T_{n-1}-T_{n-2})+\cdots+(T_2-T_1)+T_1]\\
            &=n+\frac{n}{2}+\cdots+\frac{n}{n-1}+1\\
            &=\sum_{k=1}^n\frac{n}{k}
    \end{split}
\end{equation*}
\color{black}
\subsection*{(c)}
Fix $k$ and let $A_i$ be the event that none of the first $k$ packs you buy contain coupon $i$. Find $\prob(A_1\cup A_2 \cup A_3\cup A_4)$. Then fix $\alpha>0$, take $k=\lfloor\alpha n \rfloor$ and find the limit of this probability as $n \rightarrow \infty$. Here $\lfloor x \rfloor$ denotes the largest integer $\leq x$. HINT: Consider probabilities $\prob(A_i),\prob(A_i\cap A_j)$, etc. 
\color{blue}
\\\\\textit{Solution. }By inclusion-exclusion theorem, 
\begin{equation*}
    \begin{split}
        \prob(A_1\cup A_2 \cup A_3\cup A_4) &= \sum_{i=1}^4\prob(A_i)-\sum_{i,j=4}^4\prob(A_i\cap A_j)+\sum_{i,j,k=1}^4\prob(A_i\cap A_j \cap A_k) -\prob(A_1\cap A_2\cap A_3\cap A_4)\\
            &={4 \choose 1} \left(\frac{n-1}{n}\right)^k-{4 \choose 2} \left(\frac{n-2}{n}\right)^k+{4 \choose 3} \left(\frac{n-3}{n}\right)^k-{4 \choose 4} \left(\frac{n-4}{n}\right)^k\\
            &=4 \left(1-\frac{1}{n}\right)^k-6 \left(1-\frac{2}{n}\right)^k+4 \left(1-\frac{3}{n}\right)^k- \left(1-\frac{4}{n}\right)^k
    \end{split}
\end{equation*}
Now, take $ k = \lfloor \alpha n \rfloor$ and $n \rightarrow \infty$, we have:
\begin{equation*}
    \begin{split}
        \lim_{n\rightarrow \infty}\prob(A_1\cup A_2 \cup A_3\cup A_4) &=\lim_{n\rightarrow \infty}\left(4 \left(1-\frac{1}{n}\right)^{\lfloor \alpha n \rfloor}-6 \left(1-\frac{2}{n}\right)^{\lfloor \alpha n \rfloor}+4 \left(1-\frac{3}{n}\right)^{\lfloor \alpha n \rfloor}- \left(1-\frac{4}{n}\right)^{\lfloor \alpha n \rfloor}\right)\\
            &=\lim_{n\rightarrow \infty}\left(4 \left(1-\frac{1}{n}\right)^{\alpha n}-6 \left(1-\frac{2}{n}\right)^{\alpha n}+4 \left(1-\frac{3}{n}\right)^{\alpha n}- \left(1-\frac{4}{n}\right)^{\alpha n}\right)\\
            &=4e^{-\alpha}-6e^{-2\alpha}+4e^{-3\alpha}-e^{-4\alpha}
    \end{split}
\end{equation*}
\color{black}
\subsection*{(d)}
Assume there are $n=4$ coupon types; find $\prob(T_4>k)$ for all $k\geq4$. HINT: This is short if you use what you've already done.
\color{blue}
\\\\\textit{Solution. }From (c), we have: 
\begin{equation*}
    \prob(T_4>k) = 4 \left(\frac{3}{4}\right)^k-6 \left(\frac{1}{2}\right)^k+4 \left(\frac{1}{4}\right)^k,\ k\geq 4
\end{equation*}
\color{black}
\section*{Problem 2}
Let $X$ be exponential($\lambda$) (that is, density $f(x)=\lambda e^{-\lambda x}$). The \textit{integer part} of $X$ is $\lfloor X\rfloor = \max\{k\in\mathbb{N}:k\leq X\}$. The \textit{fractional part} of $X$ is $X-\lfloor X \rfloor$. Show that $\lfloor X\rfloor$ and $X-\lfloor X\rfloor$ are independent. HINT: In general, two random variables $U,V$ are independent if the distribution of V conditioned on $U=u$ doesn't depend on $u$.
\color{blue}
\begin{proof} It suffices to show the conditional probability is same as the unconditioned one:
\begin{equation*}
\begin{split}
    \prob(\lfloor X\rfloor=n\vert X-\lfloor X\rfloor = \alpha)&= \frac{f_X(n+\alpha)}{\sum_{k=0}^\infty f_X(k+\alpha)}\\
        &=\frac{\lambda e^{-\lambda (n+\alpha)}}{\sum_{k=0}^\infty\lambda e^{-\lambda (k+\alpha)}}\\
        &=\frac{e^{-\lambda n}}{\sum_{k=0}^\infty e^{-\lambda k}}\\
        &=\frac{e^{-\lambda n}}{\frac{1}{1-e^{-\lambda}}}\\
        &=e^{-\lambda n}-e^{-\lambda(n+1)}\\
        &\stackrel{(*)}{=} \prob(n\leq X < n+1)\\
        &=\prob(\lfloor X \rfloor = n)
\end{split}
\end{equation*}
$(*)$ X is continuous.
\end{proof}
\color{black}
\section*{Problem 3}
Let $X_1,X_2,X_3$ be i.i.d. uniform in [0,1]. Let $X_{(1)}$ be the smallest of the 3 values, $X_{(2)}$ the second smallest, and $X_{(3)}$ the largest.
\subsection*{(a)}
Find the distribution function and expected value for $X_{(1)}$.
\color{blue}
\\\\\textit{Solution. } Let $F_{(i)}$ denotes the cdf of $X_{(i)}$.
\begin{equation*}
    \begin{split}
        F_{(1)}(x) &= \prob(X_{(1)} \leq x)\\
        &= 1-\prob(X_{(1)}>x)\\
        &= 1-\prod_{i=1}^3\prob(X_i>x)\\
        &= 1-(1-x)^3,\ 0<x<1
    \end{split}
\end{equation*}
Since $X_{(1)} \geq 0$, we can compute expectation using complementary cdf: 
\begin{equation*}
    \begin{split}
        \E(X_{(1)})&=\int_0^1(1-x)^3dx\\
        &=\frac{1}{4}
    \end{split}
\end{equation*}
\color{black}
\subsection*{(b)}
Find the distribution function and the density of $X_{(2)}$.
\color{blue}
\\\\\textit{Solution. }
\begin{equation*}
    \begin{split}
        \prob(X_{(2)}\leq x) &\stackrel{(*)}{=}(\prob(X_1\leq x))^3+{3 \choose 2}(\prob(X_1\leq x))^2(\prob(X_1>x))\\
            &=x^3+3x^2(1-x),\ 0<x<1\\
        f_{(2)}(x) &= 6x-6x^2,\ 0<x<1
    \end{split}
\end{equation*}

\end{document}
