\documentclass{article}
\usepackage[utf8]{inputenc}
\usepackage[english]{babel}
\usepackage{xcolor}
\usepackage{amsmath}
\usepackage{bbm}
\usepackage[]{amsthm} %lets us use \begin{proof}
\usepackage[]{amssymb} %gives us the character \varnothing
\usepackage[a4paper, total={6in, 8in}]{geometry}

\newcommand{\prob}{\mathbb{P}}

\title{MATH 505a Spring 2021 Qual Solution Attempts}
\author{Troy Tao}
\date\today

\begin{document}
\maketitle 
Contact \textcolor{blue}{yntao@usc.edu} if you think this document needs revision.

\section*{Problem 1}
A permutation $\pi$ on $n$ symbols is said to have $i$ as a fixed point if $\pi(i)=i$.
\subsection*{(a)}
Find the probability $p_n$ that a random permutation of $n$ symbols has no fixed points. HINT: Principle of inclusion and exclusion. (Your answer may involve a finite sum, which you don't need to simplify.)
\color{blue}
\newline
\newline
\textit{Solution.} Use inclusion-exclusion: let $A_i$ be the set of permutations that $\pi(i)=i$.
\begin{equation*}
    \begin{split}
        p_n &= 1- \prob(\cup_{i=1}^n A_i)\\
            &= 1 -(\sum_i\prob(A_i)-\sum_{i,j} \prob(A_i\cap A_j) +\cdots +(-1)^{n+1} \prob(\cap_{i=1}^n A_i)\\
            &= 1-n\cdot\frac{(n-1)!}{n!} +{n \choose 2}\cdot \frac{(n-2)!}{n!} +\cdots+(-1)^{n}{n \choose n} \cdot \frac{1}{n!}\\
            &= \sum_{p=0}^n(-1)^p\frac{1}{p!}
    \end{split}
\end{equation*}
\color{black}
\subsection*{(b)}
Let $S$ be a subset of $\{1,2,\cdots,n\}$ of size $k$. Find the probability that the set of fixed points of a random permutation on $n$ symbols is equal to $S$, and find the probability that a permutation has exactly $k$ fixed points. HINT: If you didn't find the values $p_j$ in part(a), you can still give answers for (b) expressed in terms of one or more $p_j$'s.
\color{blue}
\\\\
\textit{Solution.} 
\begin{equation*}
    \begin{split}
        \prob(\{\text{fixed points}\}=S) &= \prob(\pi(i)=i,\  \forall i \in S)\cdot\prob(\pi(j) \neq j,\  \forall j \in S^c|\pi(i)=i,\  \forall i \in S)\\
            &= \frac{(n-k)!}{n!}p_{n-k}\\
        \prob(\text{k fixed points}) &= {n \choose k} \cdot \prob(\{\text{fixed points}\}=S)\\
            &=\frac{p_{n-k}}{k!}
    \end{split}
\end{equation*}
We get the second probability knowing that there are ${n \choose k}$ many sets with k fixed points.
\color{black}
\subsection*{(c)}
Show that as $n$ tends to infinity, the distribution of the number of fixed points converges to a Poisson(1) distribution.
\color{blue}
\begin{proof}
\begin{equation*}
    \begin{split}
        \lim_{n\rightarrow \infty} \prob(\text{k fixed points}) &= \frac{1}{k!}\lim_{n \rightarrow \infty} \sum_{p=0}^{n-k}\frac{(-1)^p}{p!}\\
            &= \frac{1}{k!}\sum_{p=0}^{\infty}\frac{(-1)^p}{p!}\\
            &= \frac{e^{-1}}{k!}\\
            &\sim Poisson(1)
    \end{split}
\end{equation*}
\end{proof}
\color{black}
\section*{Problem 2}
Let $\{S_n,\ n\geq0\}$ be symmetric simple random walk, that is, $S_n=\sum_{i=1}^n\xi_i$ with $\xi_1,\xi_2,\cdots$ i.i.d. satisfying $\mathbb{P}(\xi_1=1)=\mathbb{P}(\xi_1=-1)=\frac{1}{2}$. Let $T=min\{n:S_n=0\}$, and write $\mathbb{P}_a$ for probabilities when the walk starts at $S_0=a$. By \textit{basic probabilities} for $\{S_n\}$ we mean probabilities of the form $\mathbb{P}_0(S_n=k)$, $\mathbb{P}_0(S_n\geq k)$, or $\mathbb{P}_0(S_n\leq k)$, all of which corresponding to starting at $S_0=0$.
\subsection*{(a)}
For $a\geq1$, $i\geq1$, $n\geq1$, express $\mathbb{P}_a(S_n=i,T\leq n)$ and $\mathbb{P}_a(S_n=i, T>n)$ in terms of finitely many basic probabilities. HINT: Reflection principle.
\color{blue}
\\\\
\textit{Solution.} Use reflection(reflect the part of path after the first approach at 0, with respect to 0), we have:
\begin{equation*}
    \prob_a(S_n=i,\ T\leq n) = \prob_a(S_n=-i) = \prob_0(S_n=i+a)
\end{equation*}
Use conditional probability, we have:
\begin{equation*}
\begin{split}
    \prob_a(S_n=i, T >n) &= \prob_a(S_n=i)\cdot\prob_a(T>n|S_n=i)\\
        &=\prob_a(S_n=i)\cdot(1-\prob_a(T\leq n|S_n=i))\\
        &=\prob_a(S_n=i)\cdot(1-\frac{\prob_a(T\leq n,S_n=i)}{\prob_a(S_n=i)})\\
        &=\prob_a(S_n=i)-\prob_a(S_n=i, T\leq n)\\
        &=\prob_0(S_n=i-a) -\prob_0(S_n=i+a)
\end{split}
\end{equation*}
\color{black}
\subsection*{(b)}
For $a\geq1$, $i\geq1$, $n\geq1$, show that
\begin{equation*}
    \mathbb{P}_a(T>n) = \sum_{j=1-a}^a\mathbb{P}_0(S_n=j).
\end{equation*}
HINT: use (a) and look for cancellation
\color{blue}
\begin{proof}
\begin{equation*}
    \begin{split}
        \prob_a(T>n) &= \sum_{i= 1}^{a+n}\prob_a(S_n=i,\ T>n)\\
            &= \sum_{i=1}^{a+n}\prob_0(S_n=i-a) -\prob_0(S_n=i+a)\\
            &= \sum_{i=1-a}^{n}\prob_0(S_n=i)-\sum_{j=1+a}^n\prob_0(S_n=j)\\
            &= \sum_{j=1-a}^a\mathbb{P}_0(S_n=j)
    \end{split}
\end{equation*}
\end{proof}
\color{black}
\subsection*{(c)}
You may take as given that $\mathbb{P}_0(S_{2m}=2j) \sim 1/\sqrt{\pi m}$ as $m \rightarrow \infty$ for each fixed $j\in\mathbb{Z}$; here $\sim$ means that ratio converges to 1. Use this to find $c,\ \alpha$ such that $\mathbb{P}_a(T>n)\sim c/n^{\alpha}$ as $n\rightarrow \infty$, where $a>0$. Does $c$ or $\alpha$ depend on $a$? HINT: It's enough to consider even $n$ - why?
\color{blue}
\begin{proof}
Assume $n$ is even where $n=2m$. For very large $n$, we have:
\begin{equation*}
    \begin{split}
        \prob_a(T>2m) &= \sum_{j=1-a}^a\mathbb{P}_0(S_{2m}=j)\\
            &= \sum_{j\in A}\mathbb{P}_0(S_{2m}=j),\ A=\{\text{even numbers in }\{1-a, 2-a, \cdots, a\}\}\\
            &\sim a\cdot \frac{1}{\sqrt{\pi m}}\\
            &= \frac{a\sqrt{\frac{2}{\pi}}}{n^{1/2}}
    \end{split}
\end{equation*}
So we get $c =a\sqrt{\frac{2}{\pi}} $ and $\alpha = \frac{1}{2}$, where $c$ depends on $a$, $\alpha$ does not. \\\\
Now we assume n is odd, and we will prove the convergence by squeezing. First by inclusion, we have the inequality:
\begin{equation*}
    \prob_a(T>n-1) \geq \prob_a(T>n) \geq \prob_a(T>n+1)
\end{equation*}
divide the expected limit:
\begin{equation*}
    \frac{\prob_a(T>n-1)}{c/n^{\alpha}} \geq \frac{\prob_a(T>n)}{c/n^{\alpha}} \geq \frac{\prob_a(T>n+1)}{c/n^{\alpha}}
\end{equation*}
normalize both sides:
\begin{equation*}
    \frac{\prob_a(T>n-1)}{c/(n-1)^{\alpha}}\cdot\left(\frac{n}{n-1}\right)^\alpha \geq \frac{\prob_a(T>n)}{c/n^{\alpha}} \geq \frac{\prob_a(T>n+1)}{c/(n+1)^{\alpha}}\cdot\left(\frac{n}{n+1}\right)^\alpha
\end{equation*}
Now, notice $n-1$ and $n+1$ are even, so if we let $n$ go to infinity, both upper and lower bound above will converge to 1.
\end{proof}
\color{black}
\section*{Problem 3}
Let $X,\ Y$ be independent standard normal $(0,1)$ random variables.
\subsection*{(a)}
Find $a$ for which $U=X+2Y$, $V=aX+Y$ are independent.
\color{blue}
\\\\
\textit{Solution. }Note that $U = (1,2)\cdot(X,Y)^T$, $V = (a,1)\cdot(X,Y)^T$, and $(X,Y)^T \sim \mathcal{N}(0,I)$. $(U,V)$ are normal vector, so $U,V$ are independent if and only if $Cov(U,V)=0$.  
\begin{equation*}
    \begin{split}
        Cov(U,V) &= (1,2)\cdot I \cdot (a,1)^T\\
            &=a+2\\
        a &= -2
    \end{split}
\end{equation*}
\color{black}
\subsection*{(b)}
Find $\mathbb{E}(XY|X+2Y=a)$ for all $a\in\mathbb{R}$. HINT: Use(a).
\color{blue}
\\\\
\textit{Solution. }Note that $X = \frac{U-2V}{5}$ and $Y = \frac{2U+V}{5}$. So the expectation turns into:
\begin{equation*}
    \frac{1}{25}\mathbb{E}(2U^2-3UV-2V^2|U=a) = \frac{1}{25}(2a^2)-3a\cdot\mathbb{E}(V)-2\cdot\mathbb{E}(V^2)) = \frac{2a^2-10}{25}
\end{equation*}
\end{document}
